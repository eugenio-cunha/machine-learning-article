\begin{resumo} 
  Nos dias atuais, há uma quantidade intensa de redações sendo produzida e 
  avaliada em vestibulares, concursos e exames. Diferentemente dos métodos 
  existentes, que processam e avaliam as redações de maneira manual, este 
  trabalho aborda uma forma automática, por meio de  aprendizagem de máquina, 
  capaz de generalizar, aprender e extrair padrões das classes de redações com 
  base no conteúdo rotulado. O método precisa de pouca intervenção humana e 
  permite a valoração de grandes quantidades de textos.  Este trabalho 
  fundamenta-se no problema de avaliação manual das competências exigidas em um 
  texto de redação do tipo dissertativo-argumentativo com temas diversificados 
  de ordem social, científica, cultural ou política. Dado um ``corpus'' de 
  redações o objetivo principal é induzir um modelo a classificar as 
  competências exigidas compondo uma nota avaliativa sobre o texto. Embasado 
  nas principais métricas de análise dos classificadores citados na literatura 
  de aprendizado de máquina, a solução proposta neste trabalho demonstrou ser 
  útil e propícia a ser utilizada em problemas que envolva a valoração de texto 
  manual por profissionais capacitados.
\end{resumo}