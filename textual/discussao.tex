\section{Discussão}

Primeiramente, como citado na \autoref{section:trabalhos_relacionados} foi 
utilizada a ferramenta de \textit{Data Mining Orange}, devido a sua boa 
classificação quanto as principais utilizadas no meio acadêmico, a capacidade 
de inferência indutiva sobre os classificadores \textit{Naive Bayes} e 
\textit{Adaboost} e por fim, por ser desenvolvida sobre a 
\textit{framework scikit-learn}, uma importante \textit{framework open-source} 
de \textit{Machine Learning} fortemente utilizada no desenvolvimento de 
aplicações \cite{scikit-learn}.

Na \autoref{subsection:pre_processamento} e explicado o uso da abordagem
\textit{bag-of-words}, apesar de inumeráveis configurações possíveis sobre este 
método, o mesmo foi utilizado em sua forma canônica, por fim, o objetivo era de 
não interfirir na assinatura do padrão encontrado no texto.

Conforme foi explicado na \autoref{subsection:validacao_cruzada} na validação 
cruzada, o \textit{dataset} balanceado foi dividido em dez partes iguais. Em 
testes anteriores foi observado, que devido a um número limitado de textos no 
\textit{dataset}, a quantidade superior as dez partições não influenciava 
diretamente os resultados das métricas de desempenho, entretanto, onerava 
consideravelmente o tempo de inferência indutiva dos classificadores.