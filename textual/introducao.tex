\newpage
\section{Introdução}

O desenvolvimento de uma redação é uma atividade prática, presente na 
cultura civilizada desde a invenção da escrita. \citeonline{lara:1995}, 
explica em seu trabalho, que com o fim da ditadura iniciou-se processo de 
redemocratização, que consequentemente, restitui a palavra ao estudante. 
O decreto 79 298, de 24 de fevereiro de 1977, definiu a volta da redação à 
escola, pela ``inclusão obrigatória da prova ou questão de redação em língua 
portuguesa'' nos concursos e vestibulares (Art. 1º, alínea d). A prova de 
redação tem sido utilizada de forma ampla em concursos, vestibulares e exames, 
tal como o ENEM, hoje o maior exame do Brasil, que no ano de 2016, contou com 
8 627 195 inscrições e a participação direta de 11 360 profissionais externos, 
na correção de 5 825 134 redações, segundo o relatório de gestão 
\citeauthor{relatorio_de_gestao:2016} (\citeyear{relatorio_de_gestao:2016}). 
Com o advento do ENEM ser um requisito para o processo seletivo de acesso às 
inúmeras universidades públicas ~\cite{sisu:2017} e a importantes programas de 
governo ~\cite{csf:2017}, este número tem aumentado, incessantemente.

Com o processamento computacional mais barato e poderoso, a crescente variedade 
e volume de dados disponíveis e o armazenamento de forma acessível, o 
Aprendizado de Máquina está no centro de muitos avanços tecnológicos, 
alcançando as áreas, antes exclusivas de seres humanos. Os carros autônomos do
projeto \citeonline{waymo:2017}, são o exemplo de uma atividade, antes 
exclusivamente humana, hoje exercida e aperfeiçoada por algoritmos de 
Aprendizado de Máquina. 

A avaliação automática de redações pode ser realizada utilizando sistemas
especialistas ou algoritmos de Aprendizado de Máquina. A primeira hipótese, 
dependente essencialmente da presença de especialistas, que detêm o 
conhecimento sobre o domínio do problema para desenvolver um conjunto de 
regras. \citeonline{negnevitsky2005artificial}, explica que o sistema 
especialista deve ser capaz de tomar suas decisões, ou seja, as regras são 
disparadas para atingir determinadas opções. Entretanto, regras desenvolvidas 
manualmente, tem um processo de manutenção e atualização complexo, o que torna 
difícil a sua utilização em diferentes domínios do problema proposto. O uso de 
algoritmos de Aprendizado de Máquina, na valoração de redações, é uma 
alternativa ao sistema especialista, exige menor esforço humano, com a simples 
abstração de extrair padrões ou características, aprender e generalizar. 

Devido à grande quantidade de redações produzidas em concursos, vestibulares 
e exames, torna-se humanamente difícil e caro, organizar e avaliar as 
competências de uma redação manualmente. A hipótese deste artigo é que um 
algoritmo de Aprendizado de Máquina pode ser útil e propício, quando utilizado 
em problemas que envolva a valoração de textos. Para avaliar e validar a 
hipótese, o método de construção do conhecimento deste estudo terá como 
fundamento o problema da recuperação de padrões, na valoração textual. Dado 
dois classificadores, com lógicas de classificação distintas, induzir ambos, a 
recuperar padrões implicitos num \textit{corpus} de redações e classificar a 
competência exigida num texto de redação em uma nova amostra, automaticamente.

Além disso, o presente estudo, com base na proposta do problema descrito, 
contribuirá na área do Aprendizado de Máquina e diretamente no processo de 
valoração de um texto em prosa do tipo dissertativo-argumentativo.