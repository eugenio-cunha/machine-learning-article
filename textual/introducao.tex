\section{Introdução}
\noindent O desenvolvimento de uma redação e uma atividade prática presente na 
cultura civilizada desde a invenção da escrita. \citeauthoronline{lara:1995} 
(\citeyear{lara:1995}) explica em seu trabalho que na década de 70 iniciou-se 
processo de redemocratização que consequentemente restitui a palavra ao 
estudante. O decreto 79.298, de 24 de fevereiro de 1977 definiu a volta da 
redação à escola pela ``inclusão obrigatória da prova ou questão de redação em 
língua portuguesa'' nos concursos e vestibulares (Art. 1º, alínea d).

A redação é aplicada no ENEM desde a primeira edição 1998, hoje o maior exame 
do Brasil, que no ano de 2016 contou com 8.627.195 escritos confirmados, e a 
participação direta de 11.360 profissionais externos na correção de 5.825.134 
redações. Com o advento do ENEM ser um requisito para o processo seletivo de 
acesso às inúmeras universidades públicas ~\cite{sisu:2017} e a importantes 
programas de governo ~\cite{csf:2017}, este número tem aumentado 
incessantemente. Segundo o edital \citeauthor{edital_enem:2016} 
(\citeyear{edital_enem:2016}), cada redação foi avaliada por, 
pelo menos, dois avaliadores, de forma independente, uma estimativa mínima de 
11.650.268 avaliações manuais das competências exigidas num texto pelo ENEM. 
Devido à grande quantidade de redações produzidas, torna-se humanamente difícil 
e caro organizar e avaliar as competências de uma redação manualmente.

Com o processamento computacional mais barato e poderoso, a crescente variedade 
e volume de dados disponíveis, e o armazenamento de forma acessível, 
o Aprendizado de Máquina está no centro de muitos avanços tecnológicos, 
alcançando as áreas antes exclusivas de seres humanos. Os carros autônomos do 
Google são o exemplo de uma atividade antes exclusivamente humana e hoje 
exercida e aperfeiçoada por algoritmos de Aprendizado de Máquina 
~\cite{waymo:2017}. O Aprendizado de Máquina está presente na nossa vida 
cotidiana como, resultados de pesquisa \textit{web}, análise de sentimento 
baseado em texto e na detecção de fraudes em operações com cartões de crédito 
entre outras aplicações ~\cite{batista1999aplicando}.

A avaliação de redações automática pode ser realizada utilizando sistemas
especialistas ou algoritmos de Aprendizado de Máquina. A primeira hipótese 
dependente essencialmente da presença de especialistas que detêm o 
conhecimento sobre o domínio do problema para desenvolver um conjunto de 
regras. O sistema especialista deve ser capaz de tomar suas decisões, ou seja, 
as regras são disparadas para atingir determinada decisão 
\cite{negnevitsky2005artificial}. Entretanto, regras desenvolvidas manualmente 
tem um processo de manutenção e atualização complexo, o que torna mais difícil 
a sua utilização em diferentes domínios do problema proposto.

O uso de algoritmos de Aprendizado de Máquina para valoração de redações é uma
alternativa ao sistema especialista, exige menor esforço humano com a 
abstração simples de extrair padrões ou características, aprender e generalizar.
Dados os benefícios, a hipótese deste artigo é que um algoritmo de Aprendizado 
de Máquina pode ser útil e propício a ser utilizada em problemas que envolva a 
valoração de texto manual por profissionais capacitados. 

Além disso, para avaliar e validar a hipótese, o método de construção do 
conhecimento deste trabalho terá como fundamento o problema de recuperação de 
padrões na valoração textual. Dado um \textit{corpus} de redações o objetivo 
principal é induzir um modelo a classificar as competências exigidas compondo 
uma nota avaliativa sobre a redação. O presente estudo com base na proposta do 
problema descrito contribuirá na área do Aprendizado de Máquina e diretamente 
no processo de valoração de um texto em prosa do tipo 
dissertativo-argumentativo.