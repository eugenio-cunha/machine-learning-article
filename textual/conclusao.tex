\section{Conclusão}

Este trabalho teve por objetivo o estudo da recuperação de padrões na valoração 
textual de redações através da classificação de textos. Foram realizadas 
extensas avaliaçõe sobre os algoritmos classificadores \textit{Naive Bayes} e 
\textit{Adaboost} no decorrer das atividades desenvolvidas para atingir os 
objetivos propostos. O classificador \textit{Adaboost} não apresentou um 
resultado tão bom quanto o observado no \textit{Naive Bayes}, mais nenhum 
momento deu-se o propósito de demonstar superioridade de um algoritimo e 
relação ao outro, mas sim, atestar a legitimidade da hipótese proposta e 
demonstrar que ambos os algoritmos quando induzidos corretamente encontraria um 
padrão implícito no texto referente a cada competência induzida no 
classificador. 

Após os métodos probabilísticos, o melhor resultado está associado ao 
classificador \textit{Naive Bayes}, portanto, considero um forte aliado para 
recuperação de padrões na valoração textual.

Para trabalhos futuros pretende-se estudar e aperfeiçoar as técnicas de 
pré-processamento e extração de atributos com o objetivo de mensurar com maior 
representatividade o padrão encontrado dentro do texto, obtendo uma melhor 
separação entre as valorações de competências.